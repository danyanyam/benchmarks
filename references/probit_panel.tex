\documentclass[11pt,a4paper]{article}
\usepackage[margin=1in]{geometry} % full-width
    \topskip        =   20pt
    \parskip        =   10pt
    \parindent      =   0 pt
    \baselineskip   =   15pt
\usepackage{amssymb, amsfonts, amsmath}
\usepackage{bm}
\usepackage{amsmath}
\usepackage{booktabs}  % neatly formatting lines
\usepackage{threeparttable}
\usepackage{graphicx}
\usepackage{caption}
\usepackage{mathtools}
\usepackage{subfig}
\usepackage[shortlabels]{enumitem}

%Russian-specific packages
%--------------------------------------
\usepackage[T2A]{fontenc}
\usepackage[utf8]{inputenc}
\usepackage[russian]{babel}

\newcommand{\bo}[1]{\mathbf{#1}}
\newcommand{\R}{\mathbb{R}}
\newcommand{\E}{\mathbb{E}}
\newcommand{\Cov}{\text{Cov}}
\newcommand{\Y}{X_i'\beta + u_i}
\newcommand{\img}[3]{
    \begin{figure*}[!hbtp]
        \centering
        \caption{#3}
        \includegraphics[scale=#2]{#1}
    \end{figure*}
}
\newcommand{\imgs}[4]{
    \begin{figure}[!hbtp]
        \centering
        \begin{minipage}{.5\textwidth}
            \centering
            \includegraphics[width=.9\linewidth]{#1}
            \captionof{figure}{#3}
        \end{minipage}
        \begin{minipage}{.49\textwidth}
            \centering
            \includegraphics[width=.9\linewidth]{#2}
            \captionof{figure}{#4}
        \end{minipage}
    \end{figure}
}

\begin{document}
\subsection*{Binary probit panel model}
\subsubsection*{Definition}
Net utility from taking action in $ t $ is
$ U_{nt} = V_{nt} + \varepsilon_{nt} $ and person takes action
if $ U_{nt}> 0 $. This is the net utility, since it is difference
between utility of taking action and that of not taking action.
The errors are correlated over time, and the covariance matrix
for $ \varepsilon_{n} = (\varepsilon_{n1}, \dots, \varepsilon_{nT}) $
is $ \Omega\in\mathbb{R}^{T\times T} $. The sequence of actions
that person $ n $ takes can be represented by a set of $ T $ dummy
variables: $ d_{nt} = 1 $ if person $ n $ took action at $ t $ and
$ d_{nt} = -1 $ otherwise. The probability of sequence if choices
$ d_n = (d_{n1}, \dots, d_{nT}) $ is
\begin{align*}
    P(d_n| V_n) & = P(U_{nt}d_{nt}>0~|V_n,~ \forall t)                                      \\
                & = P(V_{nt}d_{nt} + \varepsilon_{nt}d_{nt} > 0 ~|V_n~\forall t)            \\
                & = \int_{\varepsilon_n \in B_n} \phi_{\Omega}(\varepsilon_n)d\varepsilon_n
\end{align*}
where $ B_n $ is a set such that $ V_{nt}d_{nt} + \varepsilon_{nt}d_{nt} > 0 $
\textbf{for all} $ t $ and $ \phi_{\Omega}(\varepsilon_n)  $ is a joint
normal density with covariance matrix $ \Omega $. Once again,
mathematically:
$$ B_n = \{ \varepsilon_n = (\varepsilon_{n1}, \dots, \varepsilon_{nT}) ~|~V_{nt}d_{nt} + \varepsilon_{nt}d_{nt} > 0 \quad \forall t = 1...T\} $$

\subsubsection*{Structure on error covariance}
Structure can be placed on the covariance of the errors over time.
Further we will suppose, that the error consists of a portion that
is specific to the decision maker, reflecting his proclivity to
take the action, and a part that varies over time for each
decision maker: $ \varepsilon_{nt} = \eta_{n} + \mu_{nt} $,
where $ \mu_{nt} $  is iid over time and people with a
standard normal density, and $ \eta_n $ is iid over people
with a normal density with zero mean and variance $ \sigma $.
And $ V(\varepsilon_{nt}) = V(\eta_{n} + \mu_{nt}) = V(\eta_n) + V(\mu_{nt}) =
    \sigma +1 $. Covariance between two time periods $ s $ and $ t $
is
$$ \Cov(\varepsilon_{nt}, \varepsilon_{ns}) =
    \Cov(\eta_{n} + \mu_{nt}, \eta_{n} + \mu_{ns})= \sigma $$
Therefore covariance matrix $ \Omega $ takes form of:
$$
    \Omega = \begin{pmatrix*}
        \sigma + 1 & \sigma & \dots & \sigma \\
        \sigma  & \sigma + 1 & \dots & \sigma \\
        \dots  & \dots & \dots & \dots \\
        \sigma  & \dots & \dots & \sigma + 1 \\
    \end{pmatrix*}
$$
Only one parameter $ \sigma $, enters the covariance matrix.
its value indicates the variance in unobserved utility across
indidividuals (the variance of $ \eta_n $)
relative to the variance across time for each individual
(the variance of $ \mu_{nt} $).

\subsubsection*{Estimation of the parameters}
In order to estimate parameters we need to simulate the choice
probabilities. It is advised by \textit{Train} to use error
partioning approach. Conditional on $ \eta_n $, the probability
of not taking the action in period $ t $ is
$ P(V_{nt} + \eta_n + \mu_{nt} < 0) = P(\mu_{nt} < -(V_{nt} + \eta_n )) =
    \Phi(-(V_{nt} + \eta_n))$, where $ \Phi $ is cdf of standard normal
distribution. Probability of taking action is then:
$1 - \Phi(-(V_{nt} + \eta_n)) = \Phi(V_{nt} + \eta_n)$.
The probability of the sequence of choices $ d_n = (d_{n1}, \dots, d_{nT}) $
is then:
$$
    P(d_n |\eta_n, V_n) = \prod_{t=1}^{T}\Phi((V_{nt} + \eta_n)d_{nt})
$$
So far we have conditioned on $ \eta_n $, when in fact $ \eta_n $
is random. The unconditional probability is the integral of the
conditional probability $ P(d_n |\eta_n, V_n) $ over all possible
values of $ \eta_n $:
$$ P(d_n | V_n) = \int_{\mathbb{R}}  P(d_n |\eta_n, V_n) \phi_\sigma(\eta_n)d\eta_n$$
where $ \phi_\sigma (\eta_n) $ is normal density with zero mean and
variance $ \sigma $. Overall result:
$$
    P(d_n| V_n) = \int_{\mathbb{R}}  \prod_{t=1}^{T}\Phi((V_{nt} + \eta_n)d_{nt}) \phi_\sigma(\eta_n)d\eta_n
$$
Once again $ P(d_n| V_n) $ is unconditional (w.r.t $\eta_n$) probability that agent
$ n $ will choose sequence of choices $ d_n = (d_{n1}, \dots, d_{nT}) $,
where each action $ d_{nt} $ is either $ 1 $ if action is taken and
$-1$ otherwise. \textit{Train} (ch 5, pg. 17) proposes to use the
following algorithm for $ P(d_n| V_n) $ estimation:
\begin{enumerate}[itemsep=0em]
    \item Take a draw of $\eta_n$ from $\phi_{\sigma}(\eta_n)$.
    \item For this draw of $ \eta_n $, calculate $ P(d_n |\eta_n, V_n) $ .
    \item Repeat steps 1–2 many times, and average the results.
          This average is a simulated approximation to $ P(d_n| V_n) $ .
\end{enumerate}


\end{document}
